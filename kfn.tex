\section{KFN: Neutron Research}

\sectauthor{Tobias Richter, Simon Heybrock, Kareem Galal, European Spallation Source ERIC}

\subsection{State of federated infrastructures}

Overall the German neutron landscape consists of about 1500 researchers 
(often in university groups) and a small number of national, 
European and international
facilities. Federation happens on a 
national and European facility level. External services (for example from EOSC) play a small role.

For the users of neutrons there are a few examples of 
federated infrastructures in a operational state.
Facilities in the PaNOSC and ExPaNDS EOSC projects 
have start to open their data catalogues
to harvesting by B2Find and OpenAire via OAI-PMH. That means that information on
experimental data older than the facility specific embargo period (typically around 4 years)
is findable in federated cross-discipline metadata repositories.
In addition the two EOSC projects develop a federated domain specific search API and portal.
Another example of federation is the UmbrellaID AAI infrastructure that enables
user authentication with the same credentials in a number of participating   
computing services. 

Currently data processing capacity and storage  
is hosted by facilities in isolation. Data is often transferred away to 
local university infrastructures by the researchers and analyzed 
and compare with data from other experiments there.

\subsection{Interest in federated services}

Users of neutron facilities would benefit from:

\begin{itemize}
\item catalogue services that combine metadata from experiments carried out at different facilities and in their own labs in a consistent view
\item data sharing services with authentication to collaborate on dataset with local and external collaborators without having to move the data
\item data processing and analysis capabilities that work transparently on data irrespective of the location
\item being able to fulfil their data preservation requirements by using federated long term archiving
\end{itemize}

Availability of a 
scalable long term archiving service 
could also be welcome as a federated offering by the neutron research facilities. 
The same is true
for compute capabilities for offline analysis and simulations.