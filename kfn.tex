\section{KFN: Neutron Research}

\sectauthor{Simon Heybrock, Kareem Galal, European Spallation Source ERIC}

\subsection{Current computing and storage infrastructures}

The following computing and storage infrastructure is currently in place and in use at ESS:
Slurm batch cluster:\todo{Is the ESS using or planning to use other,
external infrastructure?}

\begin{itemize}
  \item 99 compute nodes (3312 cores)
  \item 1 GPU node
\end{itemize}

Storage infrastructure:

\begin{itemize}
  \item 0.4PB ZFS filesystem
  \item 1.7PB GPFS filesystems
\end{itemize}

\subsection{Already federated computing and storage infrastructures}

As it stands at the moment, no computing or storage infrastructure at ESS is federated.\todo{Is that the state of affairs in the whole domain?  Is there any data sharing going on already?}
All access is granted through ESS-based AAI, which is facilitated and managed by OpenLDAP and Window Active Directory services.
The only exception is the pan-learning.org website, which is a learning platform that has been migrated to the PaNOSC project, where it serves as a learning platform for the Photon and Neutron (PaN) community.
The pan-learning.org platform is configured to use the UmbrellaID AAI infrastructure, allowing users from other PaN facilities to access the online courses using their respective Research Infrastructure credentials.

\subsection{Current issues which need to be addressed}

As ESS is still in its construction phase, with no existing beamlines in operation, there are no existing operational issues for the current phase of the project.
However, one major issue that ESS is currently working on internally is unifying its Authorisation and Authentication Infrastructure (AAI) backend.
Due to historical and geographical reasons, with ESS co-hosted by both Sweden and Denmark, there is currently 2 different AAI systems in place for accessing resources at ESS, an OpenLDAP and a WindowsAD backend.
ESS is currently working on merging both systems into a single AAI
backend, allowing for a single system managing all authentication and
authorisation moving ahead.\todo{What options are you considering?
Anything that might also work for others within DIG-UM?}

\subsection{Computing and storage infrastructures we would like to federate}

Looking ahead, federating the following infrastructure would provide many benefits and ease of use to future users of ESS resources:

\begin{description}
  \item[AAI]
    Allowing non-ESS employees from other institutes and from the PaN community access in a transparent manner to ESS resources.
    This will also reduce the amount of work required by ESS to allow non-ESS users access to data and resources.
  \item[Storage]
    Federating storage backends could be beneficial for allowing inter-operability between ESS and other institutes, and for allowing easy transfer of data between the different Research Infrastructures.
    However, this requires further discussions with other institutes on which model to use, how data and storage will be shared, and how to ensure correct measures are in place to safeguard the storage infrastructure at each participating institute.\todo{do you have examples for what you'd want to safeguard against?}
  \item[Data portal (scicat)]
    The Scicat data portal allows users to search and access data from current and previous experiments.
    Using the data portal ensures data is accessible according to the FAIR principles.
    Federating this tool will ensure data is searchable and accessible across multiple research institutes, in a coordinated and simplified manner.\todo{Is this in use already?  Across the discipline?  What is it based on?  Does it already support community standards like OAI-PMH?}
\end{description}
