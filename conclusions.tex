\section{Conclusions}
\label{sect:conclusions}

\begin{itemize}
\item AAI is a common requirement.  ``Data lake'' 
\todo{Are there plausibly more?}

\item Communities can and must learn from each other.

\item ``Grid-using'' communities: what is still needed given they're
already using the same infrastructure.

\item HiLumi-LHC will be a particularly massive challenge.

\item Reliable, long-term funding for data centres is a precondition for
useful and sustainable federation.

\item This document is supposed to help in the preparation for the
upcoming Federated Infrastructure call, in particular with a view to
harmonising the proposals responding to it.\todo{Wahrscheinlich Material
für die Einleitung}

\item The complex funding structure (regional, state, federal, European
levels) requries a particularly close interaction between the players
both on the of the funders and on the side of science.

\item The federation of infrastructures needs to be informed by
requirements and constraints of other DIG-UM Topic Groups.  For
instance, an analysis workflow designed within Big Data Analytics needs
to take into account the capabilities of a federated infrastructure;
metadata generated through Research Data Management will most certainly
help implementing useful and rich archive systems.

\item going from generic (bytes) to structured (formats) to
disciplinary (data models) means decreasing ease of federation (anyone
can deal with raw storage, far fewer with, say, FITS files, and still
fewer with XYFITS from radio interferometry).

\item At the same time, the utility of federation decreases (anyone can
use raw storage, fewer people will \emph{want} to read astronomical
tables, still fewer will have reason to analyse visibilities).

\item Hence, fortunately what is most easy to federate also promises the
highest return, and that is probably what we should start with.

\item What about archive services: Do we want interoperable resource/data
collection discovery across disciplines?  And if so, why and how?  See
\url{https://github.com/msdemlei/cross-discipline-discovery} for drafts of
user stories, also Punch overarching use cases:
\url{https://www.punch4nfdi.de/use_cases/use_case_class_4____user_story/},
\url{https://www.punch4nfdi.de/use_cases/use_case_class_4/}
\end{itemize}
