\documentclass{article}

\usepackage{url}
\usepackage{graphicx}
\usepackage{xcolor}
\usepackage[utf8]{inputenc}
\usepackage{natbib}
\usepackage[a4paper,hmargin={2cm,6cm},marginparwidth=5cm,
  vmargin={2cm,3cm}]{geometry}
\usepackage{todonotes}
\bibliographystyle{agsm}
\usepackage{mathpazo}

\definecolor{ivoacolor}{rgb}{0.0,0.318,0.612}
\definecolor{linkcolor}{rgb}{0.318,0,0.318}

\RequirePackage[colorlinks,
	linkcolor=linkcolor,
	anchorcolor=linkcolor,
	citecolor=linkcolor,
	urlcolor=linkcolor,
	breaklinks=true]{hyperref}

\newcommand{\sectauthor}[1]{\marginpar{\raggedright\footnotesize by #1}}


\title{Federated Infrastructures in Research on Universe and Matter:
State of Play}
\author{DIG-UM Topic Group Federated Infrastructures}

\begin{document}
\maketitle
\begin{abstract}
As a first output of the DIG-UM Topic Group on Federated Infrastructures,
this document tries to provide a concise and necessarily subjective
overview of the state of play of digital research infrastructures in
the domains covered by DIG-UM's eight communities with a particular focus
on Germany.  Its main goal is to
help the community members to understand the practices and technologies
already established in the participating domains.  It may also be useful to
identify progress made as DIG-UM advances.

\end{abstract}

\section{Introduction}

Part of DIG-UM's mission is to improve the interoperability of the
research data infrastructures in Germany within the sectors of physics
represented through ErUM-Data's committees.  It is the purpose of this
paper to investigate what this can (or should) mean in practice.

Interoperability between data services, the necessary condition of
federation,  is of course a very desirable property from a user
perspective.  Services with common interfaces mean that users will not
have to learn new techniques when moving between service providers,
that their software continues to work as they use data and services from
different sources, quite typically also that they
get to choose between multiple implementations of a standard (e.g., in
different languages, on the local machine vs.~in the ``cloud'', etc).
Very generally, interoperability reduces lock-in to particular service
providers and thus also increases the ability of researchers to (re-)
use data from different sources.  This is why interoperability is a
pillar of the FAIR\footnote{cf.~Wilkinson et al, 2016, ``The FAIR
Guiding Principles for scientific data management and stewardship'',
Scientific Data \textbf{3}, 160018.}  principles.  An additional benefit of
interoperability is that it reduces the mental overhead on researchers
when moving from one infrastructure to another, thereby making their
research more efficient and increasing science output.

For data providers and service operators, the consideration is more
complex.  While it is true that once a standard is established and solid
server-side software support exists, their implementation is generally
simpler than a new design from the ground up.  On the other hand,
establishing standards and writing portable, re-usable software is a
major investment over a custom solution.  Where standards are not
already well established or required by funding agencies, the Principal
Investigators (PIs) of the projects that produce that data --
it is these that service providers primarily talk to in the design phase
-- may actually prefer a custom
solution, for instance because offering their data for easy consumption by
third parties is not a priority for many PIs.  Indeed, this step is
often seen as an added cost unless required by the community.

In the present case, the situation of the service operators is even more
complicated, in that their ``customers'', the researchers, will, if at
all, ask for ``vertical'' integration, i.e., interoperability with
the standards established in their field, be they formally agreed or
informally set by widely used software.  Since in both cases the
standards are set within regional or global collaborations, they are
both specific to disciplines while at the same time international in
reach.

Given that the Topic Group's mandate, along with the BMBF's 
goals, is horizontal federation, i.e., making 
infrastructures of different disciplines on the national level
interoperable and then combining them into a seamless whole.  This
concept of a Federated Science Cloud, consisting of computing,
storage, and archive resources of all eight ErUM communities in Germany,
certainly appears attractive.

To establish that, a questionaire has been provided to the participating communities
in which they were supposed to answer the following questions:


\begin{itemize}
\item What federated infrastructures are already existing which can be
used as a basis for future implementations?
\item What technologies and standards are employed?
\item What issues and challenges are interconnected with federated
infrastructures?
\item What infrastructures are planned to be federated?
\item What approaches and solutions do already exist for building
community-spanning federated infrastructures?
\end{itemize}

The information provided shall be used as a common starting point for building
a community-spanning Federated Science Cloud in Germany and to work out
a common wish list which is of interest for the majority of the communities.
A first analysis of the provided answers shows already that
a federated and interoperable authentification and authorisation infrastructure (AAI),
and a federated data infrastructure as, e.g., a Data Lake and an understanding how to 
deal with large data volumes is prioritised very highly by many ErUM-Data communities.

While good motivations for building federated infrastructures are
using synergy, potentially reducing cost, making it easier to exploit a wider range of
different resources, facilitating data sharing, optimising resource usage
by increasing the number of potential users and also the diversity of use cases,
and avoiding lock-in to specific providers,
it is also required that the communities need to develop a common understanding of:


\begin{itemize}
\item Which infrastructures are already operated in the
respective communites?  Which of them are already part of international
federations and bring along corresponding boundary conditions?

\item Which technologies are being used in federation and service
provision?  This is particularly interesting because compatible
fundamental technologies might lead the way to low-cost solutions.

\item What are the experiences with these infrastructures?  Are there
lessons learned?

\item Which further infrastructures should be made interoperable and/or
federated?

\item Are there criteria to decide when horizontal integration is
beneficial (e.g., common requirements or technologies) and when the
problems (e.g., necessarily different metadata schemes, entirely
disjunct workflows) outweigh possible benefits?

\item What is the funding model which can be applied for community overarching infrastructures ?
Science communities are typically funded to provide resources for their
own research communities, while federation of resource requires allowing
others to access the same resources in competition with your own
research community. 
\end{itemize}

The body of this paper is written by topic group members nominated for
the communities represented by the various committees, aiming for a --
necessarily subjective -- survey of the state of play in the various
disciplines.  This survey will then inform some preliminary conclusions
in sect.~\ref{sect:conclusions}.

The material here occasionally needs to be technical, and
discipline-specific jargon can not always be avoided.  We hope that the
glossary in appendix~\ref{app:glossary} will help to establish a common
background.

\input kat

\input ket

\input kfb

\input kfn

\input kfs

\input kfsi

\input khuk

\input rds

\input conclusions

\appendix

\input glossary
\end{document}
