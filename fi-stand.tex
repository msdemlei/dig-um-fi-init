\documentclass{article}

\usepackage{url}
\usepackage{graphicx}
\usepackage{xcolor}
\usepackage[utf8]{inputenc}
\usepackage{natbib}
\usepackage[a4paper,hmargin={2cm,6cm},marginparwidth=5cm,
  vmargin={2cm,3cm}]{geometry}
\usepackage{todonotes}
\bibliographystyle{agsm}
\usepackage{mathpazo}

\definecolor{ivoacolor}{rgb}{0.0,0.318,0.612}
\definecolor{linkcolor}{rgb}{0.318,0,0.318}

\RequirePackage[colorlinks,
	linkcolor=linkcolor,
	anchorcolor=linkcolor,
	citecolor=linkcolor,
	urlcolor=linkcolor,
	breaklinks=true]{hyperref}

\newcommand{\sectauthor}[1]{\marginpar{\raggedright\footnotesize by #1}}


\title{Federated Infrastructures in Research on Universe and Matter:
State of Play}
\author{DIG-UM Topic Group Federated Infrastructures}

\begin{document}
\maketitle
\begin{abstract}
As a first output of the DIG-UM Topic Group on Federated Infrastructures,
this document tries to provide a concise and necessarily subjective
overview of the state of play of digital research infrastructures in
the domains covered by DIG-UM's eight committees.  Its main goal is to
help the comittee members understand the practices and technologies
already established in the other domains.  It may also be useful to
identify progress made as DIG-UM progresses.

\end{abstract}

\section{Introduction}

Part of DIG-UM's mission is to improve the interoperability of the
research data infrastructures in Germany within the sectors of physics
represented through ErUM-Data's committees.  It is the purpose of this
paper to investigate what this can (or should) mean in practice.

Interoperability between data services, the necessary condition of
federation,  is of course a very desirable property from a user
perspective.  Services with common interfaces mean that users will not
have to learn new techniques when moving between service providers,
that their software continues to work, quite typicially also that they
get to choose between multiple implementations of a standard (e.g., in
different languages, on the local machine vs.~in the ``cloud'', etc).
Very generally, interoperability reduces lock-in to particular service
providers and thus also increases the ability of researchers to (re-)
use data from different sources.  This is why interoperability is a
pillar of the FAIR principles.  An additional benefit of
interoperability is that it reduces the mental overhead on researchers
when moving from one infrastructure to another, thereby making their
research more efficient and increasing science output. 

For data providers and service operators, the consideration is more
complex.  While it is true that once a standard is established and solid
server-side software support exists, their implementation is generally
simpler than a new design from the ground up.  On the other hand,
establishing standards and writing portable, re-usable software is a
major investment over a custom solution.  Where standards are not
already well established or required by funding agencies, the PIs that
data providers usually primarily talk to may actually prefer a custom
solution, if only because offering their data for easy consumption by
third parties is not a priority for many PIs.  Indeed, this step is
often seen as an added cost unless required by the community (eg: PDB
deposition for protein structures).

In the present case, the situation of the service operators is even more
complicated, in that their ``customers'', the researchers, will, if at
all, ask for ``vertical'' integration, i.e., interoperability with
the standards established in their field, be they formally agreed or
informally set by widely used software.  Since in both cases the
standards are set within regional or global collaborations, they are
both specific to disciplinnes while at the same time international in
reach.

Against that, DIG-UM's mandate, along with the BMBF's purview, is 
horizontal federation, i.e., providing interoperability between
infrastructures of different disciplines on the national level.

In particular the tension between the somewhat natural vertical
integration -- that has been going on for decades in many disciplines --
and the horizontal integration resulting from a political mandate makes
it very desirable to establish, at the beginning of DIG-UM's activities:

\begin{itemize}
\item Which infrastructures are already operated in the
respective communites?  Which of them are already part of international,
vertical federations (and thus perhaps less malleable)?

\item Which technologies are being used in federation and service
provision?  This is particularly interesting because compatible
fundamental technologies might lead the way to particularly low-hanging
fruit.

\item What are the experiences with these infrastructures?  Are there
lessons learned?

\item Which further infrastructures should be made interoperable and/or
federated, whether horizontally or vertically?

\item What do we want to enable by fostering horizontal (i.e., on the
national level) integration?  Can we write stories showing the
benefits this will bring to our current or future users?

\item Are there criteria to decide when horizontal integration is
beneficial (e.g., common requirements or technologies) and when the
problems (e.g., necessarily different metadata schemes, entirely
disjunct workflows) outweigh possible benefits?

\item What is the funding model, or more precisely the incentive?
Science communities are typically funded to provide resources for their
own research communities, while federation of resource requires allowing
others to access the same resources in competition with your own
research community. At its most basic -- what is the incentive to
provide federated resources to others you are not funded to support?
\end{itemize}

The body of this paper is written by topic group members nominated for
the communities represented by the various committees, trying a --
necessarily subjective -- survey of the state of play in the various
disciplines.  This survey will then inform some preliminary conclusions
in sect.~\ref{sect:conclusions}.

The material here occasionally needs to be technical, and
discipline-specific jargon can not always be avoided.  We hope the
glossary in appendix~\ref{app:glossary} will help establish a common
background.

\input kat

\input ket

\input kfb

\input kfn

\input kfs

\input kfsi

\input khuk

\input rds

\input conclusions

\appendix

\input glossary
\end{document}
