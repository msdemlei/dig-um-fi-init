\section{Glossary}
\label{app:glossary}

\begin{description}
\item[CADC] Canadian Astronomy Data Centre, a Victoria, BC-based
institution managing data publication for the Canadian astronomy
community (and a good deal beyond that) that GAVO would consider a model
for how such a thing should be organised.

\item[CDS] Centre de Donn\'ees astronomique de Strasbourg, a French data
centre that keeps and curates must astronomical data published in
tabular form.

\item[CTA] Cherenkov Telescope Array.

\item[CVMFS] CERN Virtual Machine File System, a technology to distribute a central software installation globally via a squid caches.

\item[DataCite] A technology for minting persistent identifiers for data
artefacts that comes with a minimal and cross-disciplinary metadata
schema for describing data resources and relationships between them.

\item[EGI] European Grid Initiative.

\item[ESA] European Space Agency.

\item[GAVO] German Astrophysical Virtual Observatory, the German
contribution to the global VO effort.

\item[GPFS] General Purpose File System, an IBM-developed cluster file
system.

\item[IETF] Internet Engineering Task Force, the body that manages the
development and evolution of the internet's core standards.

\item[IVOA] International Virtual Observatory Alliance, the body that
manages the development and evolution of standards in the VO.

\item[LHC] Large Hadron Collider, a 20\,km long circular particle collider operated at CERN.
 
\item[MAST] see STScI.

\item[OAI-PMH] Open Archives Initiative Protocol for Metadata
Harvesting, a widely-deployed mechanism to exchange resource metadata
allowing incremental harvesting.

\item[OSG] Open Science Grid, a Grid initiative of US resource providers. OSG provides a middleware distribution that is also used elsewhere, primarily in America.

\item[PI] Principal Investigator; used here to designate the researchers
producing data and/or consuming services, typcially within a specific
project of limited duration.

\item[RDBMS] Relational Database Management System, a class of software
allowing efficient querying and manipulation of tabular data.

\item[RDF] Resource Description Framework, a W3C standard to describe
semantic resources like vocabularies and ontologies.

\item[SQL] Structured Query Language, the de-facto standard for writing
queries against RDBMSes.

\item[SSO] Single Sign On, a group of technologies that usually features
a single service authenticating users and granting them tokens giving
access to many other access-controlled resources.

\item[STScI] Space Telescope Science Institute, a Baltimore, MD-based
facility publishing data from several NASA instruments, e.g., through
its MAST archive.

\item[TAP] Table Access Protocol, a VO standard letting users execute
queries in a SQL-like standard language on remote servers
(cf.~sect.~\ref{sect:rds}).

\item[UML] Unified Modeling Language, a set of technologies and standards
designed to facilitate reasoning about software and its development.

\item[VO] Depending on context, this could be Virtual Observatory (cf.
sect.~\ref{sect:rds}) or Virtual Organisation, a concept in the Grid's
authorisation infrastructure.

\item[W3C] World Wide Web Consortium, the body that manages the
evolution of Web standards like HTML, XML, CSS, RDF, and so on.

\item[WLCG] Word Wide LHC Computing Grid, a collaboration of resource providers to support the resource needs of the LHC experiments.

\end{description}
