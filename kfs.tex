\section{KFS: Synchrotron Radiation}

\sectauthor{Anton Barty, DESY}

The photon science large scale facilities maintain largely separate
computing infrastructure located and maintained on-site at each of the
facilities.   This is largely historical as computing services have
grown locally as detector demands increased, linked to local data
acquisition and driven by separate funding streams for each of the
facilities.  Computing has been seen as a part of the local facility
infrastructure rather than a federated service across infrastructures.
Each facility has its own hardware, login and job scheduling systems. 


That said – DESY and the European XFEL host their hardware in the same
data centre at DESY and have some ability to share resources.   The
compute and storage systems are financed separately and are logically
separated in the data centre even though they and share AAI
infrastructure, file transfer services, tape archiving systems and the
like.  Both run on GPFS\todo{gloss?} for high speed storage and use dCache to manage
tape archiving, use the same common login authentication and remote
access gateways. Locate compute nodes within the shared Maxwell cluster.
Opportunistic use of free CPU resources is possible as data from one
facility can be processed on nodes owned by the other facility due to
shared AAI and GPFS infrastructure; however, each facility gets priority
over its own resources when needed.  The arrangement is better described
as symbiotic co-existence than a true federation of resources.\todo{This
sounds as if it weren't terribly far from KET in terms of tech.  Is that
true?}


The resources provided to the user community at their home institutions
(Universities, MPG, industry) are definitely not federated. 
